\documentclass[12pt]{article}

\usepackage{amsfonts,amssymb}
\usepackage[utf8]{inputenc}
\usepackage[russian]{babel}
\usepackage[dvips]{graphicx}
\usepackage{amsmath}
\usepackage{amsfonts}
\usepackage[ruled, lined]{algorithm2e}

\textheight=220mm
\textwidth=160mm

\newcommand{\sgn}{\operatorname{sgn}}
\newcommand{\argmax}{\operatorname{argmax}}
\newcommand{\NA}{\operatorname{NA}}
\newcommand{\OR}{\operatorname{ or }}
\newcommand{\LCS}{\operatorname{LCS}}
%\DeclareMathOperator{\sgn}{sgn}

\title{\bf Домашнее задание по курсу \\ <<Методы
и структуры данных поиска.>>}
\author{А.Е. Оразаев}
\date{}
\begin{document}

\voffset=-20mm
\hoffset=-12mm
\font\Got=eufm10 scaled\magstep2 \font\Got=eufm10

\maketitle

\section{Задача 1-1.}
\paragraph{Описание алгоритма.}
Условимся, что будем в дальейшем индексировать строки начиная с 0,
а не с 1, дабы быть ближе к реализации. В таком случае запишем определение
префикс функции для $ s $: для каждого $ i \in [0, |s| - 1] $:
$$
    p[i] = \max\{j: 0 \le j < i, s[0 .. j - 1] = s[i - j + 1 .. i]\}
$$

Итак у нас есть строка $ s $ из $ 0 < N \le 1000000$ символов.
Заведем вектор $ p $ длины N для хранения значений префикс функции.
Проинициализируем $ p[0] = 0 $.

Далее для всех $ i $ в цикле от $ 1 $ до $ N - 1 $ выполняем следующие шаги:
\begin{enumerate}
    \item Положим $ j = p[i - 1] $.
    \item Если $ s[j] == s[i] $, то $ p[i] = j + 1 $, переходим
          к следующему $ i $.
    \item Если $ j \ne 0 $, то $ j = p[j - 1] $ и переходим ко 2-ому шагу.
    \item В противном случае $ j = 0 $, пологаем $ p[i] = 0 $ и далее
          переходим к следующему $ i $.
\end{enumerate}

Таким образом считаем префикс функцию для всей строки $ s $.

% Controll example: FALSE
% s: abxzzabxhhhhabxzzabxk......................abxzzabxhhhhabxzzabxz
% p: 000001230000123456780......................123456789012345678904
%                                               000000000111111111120


\paragraph{Доказательство.}
Доказательство приведено во 2-ой лекции.

% FIXME: write proof
% Докажем что $ p[i + 1] \le p[i] + 1 $.
% Допустим, что $ p[i + 1] > p[i] + 1 $, тогда
% $ s[0 .. p[i + 1] - 1] = s[(i + 1) - p[i + 1] + 1.. i + 1] $, отсюда можно
% сказать, что $ s[0 .. (p[i + 1] - 1) - 1] = s[i - (p[i + 1] - 1) + 1 .. i] $,
% а следовательно $ p[i] \ge p[i + 1] - 1 $ или $ p[i + 1] \le p[i] + 1 $,
% противоречие.
% 
% Ну это и понятно из чисто интуитивных соображений, очередной префикс в $ p[i + 1] $,
% не может быть больше $ p[i] + 1$, так как новый, единственный символ может
% внести лишь вклад равный еденице в длину нового префикса.
% 
% В случае если $ s[i + 1] \ne s[p[i]] $, тогда можно говорить, что префикс $ p[i + 1] $
% не может быть длиннее чем $ p[p[i] - 1] + 1 $, так как cледующий максимальный по длинне
% префикс к которому может присоедениться $ s[i + 1] $, это $ p[p[i] - 1 ] $ уже посчитанный
% нами ранее. Если $ p[i] \ne 0$, конечно, в этом случае очевидно, что префикса к которому
% можно присоеденить $ s[i + 1] $ просто нет.






\paragraph{Сложность}
% FIXME: write complexity
Сложность по времений $ O(N) $.
Сложность по памяти $ O(N) $.

\end{document}
