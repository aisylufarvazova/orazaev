\documentclass[12pt]{article}

\usepackage{amsfonts,amssymb}
\usepackage[utf8]{inputenc}
\usepackage[russian]{babel}
\usepackage[dvips]{graphicx}
\usepackage{amsmath}
\usepackage{amsfonts}

\textheight=220mm
\textwidth=160mm

\newcommand{\sgn}{\operatorname{sgn}}
\newcommand{\argmax}{\operatorname{argmax}}
\newcommand{\NA}{\operatorname{NA}}
\newcommand{\OR}{\operatorname{ or }}
\newcommand{\LCS}{\operatorname{LCS}}
%\DeclareMathOperator{\sgn}{sgn}

\title{\bf Домашнее задание по курсу \\ <<Методы
и структуры данных поиска.>>}
\author{А.Е. Оразаев}
\date{}
\begin{document}

\voffset=-20mm
\hoffset=-12mm
\font\Got=eufm10 scaled\magstep2 \font\Got=eufm10

\maketitle

\section{Задача 3-4.}
\paragraph{Описание алгоритма.}
Очевидно, что задача сводится к задаче поддержки k-ой статистики в очереди.
Операциям $ R $ и $ L $ будут соответствовать операции push и pop
соответственно.

Для поддержки k-ой статистики будем хранить отсортированный список $ S $
элементов между $ l $ и $ r $. Вместо очереди будем хранить очередь
итераторов $ I $ на элементы $ S $ в порядке их добавления с помощью команд
$ R $ и $ L $. А также, чтобы каждый раз не бегать к k-ому элементу $ S $
будем хранить итератор $ it_k $ на него.

В начале поместим в S только один первый элемент входной полседовательности,
на котором изначально установлены итераторы $ l $ и $ r $, а $ it_k $
присвоим значение $ S.end() $, в случае если $ k \ne 1 $ и $ S.begin() $,
в противном случае.

Далее пока не кончились команды выполняем следующие действия:
\begin{enumerate}
    \item Если встретилась команда $ R $:
        \begin{enumerate}
            \item Читаем из входной последовательности элемент $ elem $.
            \item Пробегаем по списку $ S $, пока не найдем первый элемент
                  $ s_i $, такой что $ s_i -ge elem $.
            \item Вставляем элемент $ elem $ перед элементом $ s_i $. А
                  итератор на новый элемент $ s_i $, который теперь равен
                  $ elem $, помещаем в конец $ I $.
            \item Если $ S.size() \ge k $ и $ it_k = S.end() $ или
                  $ elem \le *it_k $, то $ it_k -= 1$.
        \end{enumerate}
    \item Если встретилась команда $ L $:
        \begin{enumerate}
            \item Извлекаем первый элемент $ it $ из $ I $.
            \item Если $ it != S.end() $ и $ it == it_k $ или
                  $ *it < *it_k $, то ++$ it_k $.
            \item Удаляем $ it $ из $ S $.
        \end{enumerate}
    \item Если $ it_k != S.end() $ выводим $ *it_k $ на экран, иначе
          выводим $ -1 $.
\end{enumerate}



\paragraph{Доказательство.}
Во всех последующих рассуждениях подразумевается тот факт, что $ S $ всегда
отсортирован и операциями проводимыми с $ S $ это поддерживается.

То что операция $ L $ поддерживает отсортированность $ S $, очевидно. При
выполнении $ R $ вставка элемента $ elem $ выполняется после самого
правого элемента меньше чем $ elem $, что очевидно оставляет $ S $
отсортированным.

Таким образом главное правильно позиционировать итератор на k-ый элемент
$ S $ в случае если $ S.size() \ge k $, и на $ S.end() $ в противном случае.
Рассмотрим все вомзможные случаи:

Рассмотрим команду $ L $.
\begin{enumerate}
    \item Если $ S.size() <  k $, то все работает верно, так как k-ого
          элемента просто нет и не появится, так как количество элементов
          уменьшается.
    \item Если $ S.size() \ge k $, то в случае, если извлекается элемент
          $ it_k $ или элемент меньше чем $ it_k $, то новый k-ый элемент
          будет следующим после $ it_k $, либо если после выполнения стало
          $ S.size() < k $. В обоих случая модель с перемещением итератора
          на следующую позицию работает верно так как в первом случае
          итератор переместится на следующую позицию, а во втором на
          $ S.end() $ соответственно.
    \item Если $ S.size() \ge k $, то в случае, если извлекается элемент
          $ S[j] $, где $ j > k $, то k-ый элемент не изменится, и алгоритм
          не изменит сам итератор, что верно.
\end{enumerate}

Рассмотрим команду $ R $.
\begin{enumerate}
    \item Если после добавления элемента $ S.size() < k $, то все работает
          верно, так как k-ого элемента нет, итератор останется на
          $ S.end() $, на экран попадет $ -1 $.
    \item Если после добавления элемента $ elem $ $ S.size() \ge k $,
          и если $ elem $ оказался меньше либо равным одному из превых
          k-элементов, то он встанет после последнего элемента $ S[j] $,
          такого, что $ S[j] < elem $, то все элементы $ S[j + 1] \dots
          S.back() $ сдвинутся вправо на один элемент, и следовательно
          бывший k-ый элемент также сдвинется дальше на один элемент,
          поэтому мы должны сдвинуть итератор влево на один элемент, что
          и делает алгоритм.
    \item Если после добавления элемента $ elem $ $ S.size() \ge k $,
          и если $ elem $ оказался больше k-ого элемента, а следовательно
          и больше чем первые k элементов, так как S отсортирован и это
          поддерживается операциями с ним, то $ elem $ встанет после
          первых $ k $ элементов не изменив k-ой статистики, таким образом
          мы должны оставить итератор на своем месте, как и делает алгоритм.
\end{enumerate}

Таким образом итератор на k-ый элемент отсортированного $ S $ позиционируется
верно, следовательно мы верно выводим k-ую статистику обращаясь к этому
элементу.


\paragraph{Сложность}
Сложность команды $ R $ $ O(n) $ и $ n $ в худшем случае.
Сложность $ L $ всегда $ O(1) $.

В итоге сложность по времени $ O(n^2) $, алгоритм требует $ O(n) $
дополнительной памяти.

\end{document}
