\documentclass[12pt]{article}

\usepackage{amsfonts,amssymb}
\usepackage[utf8]{inputenc}
\usepackage[russian]{babel}
\usepackage[dvips]{graphicx}
\usepackage{amsmath}

\textheight=220mm
\textwidth=160mm

\title{\bf Домашнее задание по курсу \\ <<Комбинаторика 
и теория вероятностей>>}
\author{А.Е. Оразаев}
\date{}
\begin{document}

\voffset=-20mm 
\hoffset=-12mm
\font\Got=eufm10 scaled\magstep2 \font\Got=eufm10

\maketitle

\section{Стандартные задачи}



\paragraph{Задача 1. 1 балл.} В языке одного древнего племени было 
6 гласных и 8 согласных, причем при составлении слов гласные и 
согласные непременно чередовались. Сколько слов из девяти букв 
могло быть в этом языке? 

\paragraph{\bf Решение:}
Учитывая, что буквы должны чередоваться, то составить слово из 9 букв
можно из 5 согласных и 4 гласных, а также из 4 согласных и 5 гласных.
Таким образом возможны только 2 типа слов: <<СГСГСГСГС>> и <<ГСГСГСГСГ>>, где
Г -- гласная, а С -- согласная буква.

Таким образом, количество слов типа <<СГСГСГСГС>> $ W_1 = 6^5 * 8^4 $,
а слов типа <<ГСГСГСГСГ>> $ W_2 = 6^4 * 8^5 $. А всего слов из 9 букв:
$$ ANS = W_1 + W_2 = 6^5 * 8^4 + 6^4 * 8^5 = 74317824 \text{ слова}$$



\paragraph{Задача 2. 2 балла.} Пассажир оставил вещи в 
автоматической камере хранения, а, когда пришел получать вещи, 
выяснилось, что он забыл номер. Он только помнил, что в номере 
были числа 23 и 37. Чтобы открыть камеру, нужно правильно 
набрать пятизначный номер. Какое наименьшее количество номеров 
нужно перебрать, чтобы наверняка открыть камеру? (Числа 23 и 37 
можно увидеть и в числе 237.)

\paragraph{\bf Решение:}
Рассмотрим возможные вариации кода: \\
В случае <<237>>:
\begin{enumerate}
\item 237**
\item *237*
\item **237
\end{enumerate}
В случае <<23 и 37>>:
\begin{enumerate}
\item 23*37
\item *2337
\item 2337*
\end{enumerate}
Последний случай надо будет домножать на 2, потому что числа 23 и 37
можно поменять местами.

У нас получается $ K_1 = 3 * 10^2 $ кодов для случая <<237>>, а также
$ K_2 = 3 * 10^1 $ кодов для случая <<23 и 37>>. А всего кодов, которые
стоит перебирать:
$$ ANS = K_1 + 2 * K_2 = 3 * 10^2 + 2 * 3 * 10^1 = 360 $$



\paragraph{Задача 3.} Двадцать школьников пришли в столовую. 
Сколькими различными способами они могут выстроиться в очередь, если:
а) (1 балл) Сережа обязательно стоит первым, Андрей -- четвертым, а Володя -- 
седьмым? б) (1 балл) Сережа, Андрей и Володя непременно
занимают первую, четвертую и седьмую позиции в очереди (в произвольном
порядке)? в) (1 балл) Сережа, Андрей и Володя занимают позиции с номерами 
$ i,j,k $ (опять-таки, в произвольном порядке), причем 
$ j-i = 3 $, $ k-j=3 $? 

\paragraph{\bf Решение:}
a) Если 3 школьника надо стоят на определенных местах, то осталось разместить
оставшихся в оставшиеся места:
$$ ANS_a = A_{17}^{17} = 17! = 355687428096000 $$
б) Теперь можно перемешать Андрея, Сережу и Володю на их местах. 
Пользуемся правилом умножения и предыдущим ответом:
$$ ANS_b = A_{3}^{3} * ANS_a = 2134124568576000 $$
в) Исходя из условия $ i \in [1; 14] $, то есть у нас 15 вариантов мест, для
знакомых нам школьников. И снова пользуемся правилом умножения:
$$ ANS_c = 15 * ANS_b = 32011868528640000 $$



\paragraph{Задача 4. 1 балл.} Сколькими способами можно выбрать четырех 
человек на четыре должности, если имеется девять кандидатов на эти 
должности? 

\paragraph{\bf Решение:}
Будем считать, что все должности разные, так как обратное не оговорено. Следовательно
речь идет о размещении 9 человек на 4 должности:
$$ ANS = A_{9}^{4} = \frac{9!}{5!} = 3024 $$


\paragraph{Задача 5.} Из класса, в котором учатся 28 человек, 
назначаются на дежурство в столовую 4 человека. а) (1 балл) Сколькими способами 
это можно сделать? б) (1 балл) Сколько существует способов набрать команду 
дежурных, в которую попадет ученик этого класса Коля Васин?

\paragraph{\bf Решение:}
а) Так как одна <<должность>> дежурного не отличается от другой, следовательно
разговор идет о 4-сочетании из 28 человек:
$$ ANS_a = C_{28}^{4} = \frac{28!}{4! * (28 - 4)!} = 20475 $$
б) Так как один ученик определен, то речь пойдет о 3-сочетании из 27 человек:
$$ ANS_b = C_{27}^{3} = \frac{27!}{3! * (27 - 3)!} = 2925 $$

\paragraph{Задача 6. 1 балл.} В кондитерском магазине продавались 4 сорта 
пирожных: наполеоны, эклеры, песочные и слоеные. Сколькими способами 
можно купить 7 пирожных? 

\paragraph{\bf Решение:}
Посмотрим на задачу с другого ракурса, а точнее закодируем ее с помощью
нулей и едениц. В начало строки запишем столько едениц, сколько купили наполеонов, затем,
чтобы отделить наполеоны от остальных, запишем 0, далее запишем столько едениц
сколько купили эклеров, и снова пишем 0... и так далее кодируем все купленные <<печеньки>>.

В итоге мы переформулируем текущую задачу в следующую: <<Сколькими способами можно
расставить 7 едениц и 3 нуля?>> Ответом будет:
$$ ANS = P(7, 3) = \frac{10!}{7! * 3!} = 120 $$
П.С. Только приступив к решению следующей задачи, я понял, что лишний раз доказал формулу
сочетаний с повторениями.

\paragraph{Задача 7. 2 балла.} Сколькими способами можно выбрать четыре 
различных 7-со\-че\-та\-ния с повторениями из множества букв русского 
алфавита?

\paragraph{\bf Решение:}
Количество 7-сочетаний с повторениями в множестве букв русского алфавита
будет равняться $ W_7 = \overline{C}_{33}^{7} = C_{39}^{7} = 15380937 $. Из этого
количества <<7-буквенных слов>> на предстоит выбрать 4:
$$ ANS = C_{W_7}^{4} = C_{15380937}^{4} = 2331952833639264081719371470 $$

\paragraph{Задача 8. 3 балла.} Сколькими способами можно так выбрать 
четыре различных 7-со\-че\-та\-ния с повторениями из множества букв русского 
алфавита, чтобы в каждом из этих сочетаний присутствовала хотя бы 
одна буква из набора $ \{\text{ж},\text{а},\text{б}\} $? 

\paragraph{\bf Решение:}
Для начала посчитаем сколько же всего 7-сочетаний с повторениями без букв из набора
$ \{\text{ж},\text{а},\text{б}\} $, это очевидно: $ \overline{C}_{30}^{7} = C_{36}^{7} $
Теперь отнимем полученное значение от общего числа 7-сочетаний с повторениями и получим:
$ N = \overline{C}_{33}^{7} - C_{36}^{7} = C_{39}^{7} - C_{36}^{7} = 7033257$
И для того чтобы получить ответ выберем из этого количества 4 элемента:
$$ ANS = C_N^4 = C_{7033257}^4 = 101956363330287044260745910 $$

\paragraph{Задача 9.} На клетчатой бумаге изображен квадрат, 
каждая сторона которого умещает ровно $ n $ клеток. Сколько 
в этом квадрате можно нарисовать различных а) (2 балла) квадратиков? б)
(2 балла)
прямоугольников? в) (3 балла) букв "г" (в том числе и как угодно перевернутых)? 

\paragraph{\bf Решение:}
a) Для начала уточним, что мы будем понимать под <<отношением одинаковости>>. Интуинтивно
2 квадрата являются одинаковыми, если у них одинаковая длинна стороны. Из этого очевидно,
что число разных квадратиков, которых можно нарисовать в нашем большом квадрате:
$$ ANS_a^{(1)} = n $$
Тем не менее, так задача на 2 балла выглядит как-то слишком просто, поэтому рассмотрим
случай, когда два одинаковых(по принятому соглашению) квадратика на разных позициях в большом
квадрате считаются <<различными>>. В таком случае в квадрате $n \times n$ мы можем начертить $(k + 1)^2$ <<различных>>
квадратиков размера $ (n-k) \times (n-k)$. И уже здесь легко проглядывается сумма квадратов:
$$ ANS_a^{(2)} = \sum\limits_{k = 0}^{n-1} (k + 1)^2 = \frac{n(n+0.5)(n+1)}{3} $$
б) Опять же сначала рассмотрим случай где одинаковые прямоугольники на разных позициях не считаются различными.
Тогда легко заметить, что в квадрате $1 \times 1$ можно начертить $R(1) = 1$ прямоугольник, а в квадрате
$n \times n$ можно начертить $R(n) = R(n - 1) + 3$. Это рекурентное соотношение, очевидно, разрешается вот так:
$$ ANS_b^{(1)} = 3(n - 1) + 1 $$
Рассмотрим второй случай. Теперь в квадрате $ n \times n $ можно начертить $ (k+1)(j+1) $ прямоугольников размера
$(n - k) \times (n - j) $. Суммируем все это дело:
$$ ANS_b^{(2)} = \sum\limits_{k, j = 1}^{n} k \cdot j = \sum\limits_{j = 1}^n j \cdot \sum\limits_{k = 1}^n k =
\left(\frac{n(n + 1)}{2}\right)^2 $$
в) Для начала будем считать, что букву крутить нельзя. Очевидно, что букву <<Г>> можно начертить только в 
прямоугольнике размером больше либо равном $ 2 \times 2 $. Следовательно количество букв <<Г>> будет равняться 
количеству прямоугольников размером $ >= 2 \times 2 $. Чтобы учесть вращение -- просто умножим полученное
число на 4. Итак используя решение задачи <<б>> запишем сразу 2 ответа, для 2-х уже знакомых нам случаев:
$$ 
ANS_c^{(1)} = \left\{
\begin{array}{rl}
4(3(n - 1) + 1 - 3) = 12(n - 1) - 8 &\mbox{,}\ n > 1\\
0 &\mbox{,}\ n = 0, 1
\end{array}
\right.
$$
$$
ANS_c^{(2)} = 
4\left(\left(\frac{n(n + 1)}{2}\right)^2 - 2n(n - 1) - n^2\right) = (n^2 + n)^2 - 12 n^2 - 8 n
$$

\paragraph{Задача 10. 2 балла.} В прямоугольнике площади 5 расположены 9 фигур 
площади 1 каждая. Докажите, что найдутся две фигуры, площадь общей 
части которых не меньше $ \frac{1}{9} $. 

\paragraph{\bf Решение:}
Доказательство от противного. Пусть $ S = 5 $ -- площадь нашего прямоугольника, а также $ S_{i_1, ..., i_k} $ -- площадь пересечения
фигур $ i_1, ..., i_k $, где $ i_j < 9 \text{ и } i_j = i_c \Leftrightarrow j = c $. Тогда
$$ S - \sum\limits_{1 \le i \le 9} S_i + \sum\limits_{1 \le i_1, i_2 \le 9} S_{i_1, i_2} - ... - S_{1,2,3,4,5,6,7,8,9} \ge 0 $$.
И так как $\sum S_{i_1, ..., i_k} \ge \sum S_{i_1, ..., i_j} \Leftrightarrow k \le j$, следовательно:
$$
\sum\limits_{1 \le i_1, i_2 \le 9} S_{i_1, i_2} \ge \sum\limits_{1 \le i \le 9} S_i - S = 4
$$
В левой сумме $ C_9^2 = 9\cdot4 $ челенов которые строго меньше $ \frac{1}{9} $, 
следоветельно эта сумма строго меньше 4, что противоречит изложенным выше выкладкам.

\end{document}
