\documentclass[12pt]{article}

\usepackage{amsfonts,amssymb}
\usepackage[utf8]{inputenc}
\usepackage[russian]{babel}
\usepackage[dvips]{graphicx}

\textheight=220mm
\textwidth=160mm

\title{\bf Домашнее задание по курсу \\ <<Комбинаторика 
и теория вероятностей>>}
\author{А.Е. Оразаев}
\date{}
\begin{document}

\voffset=-20mm 
\hoffset=-12mm
\font\Got=eufm10 scaled\magstep2 \font\Got=eufm10

\maketitle

\section{Стандартные задачи}



\paragraph{Задача 1. 1 балл.} В языке одного древнего племени было 
6 гласных и 8 согласных, причем при составлении слов гласные и 
согласные непременно чередовались. Сколько слов из девяти букв 
могло быть в этом языке? 

\paragraph{\bf Решение:}
Учитывая, что буквы должны чередоваться, то составить слово из 9 букв
можно из 5 согласных и 4 гласных, а также из 4 согласных и 5 гласных.
Таким образом возможны только 2 типа слов: <<СГСГСГСГС>> и <<ГСГСГСГСГ>>, где
Г -- гласная, а С -- согласная буква.

Таким образом, количество слов типа <<СГСГСГСГС>> $ W_1 = 6^5 * 8^4 $,
а слов типа <<ГСГСГСГСГ>> $ W_2 = 6^4 * 8^5 $. А всего слов из 9 букв:
$$ W = W_1 + W_2 = 6^5 * 8^4 + 6^4 * 8^5 = 74317824 слова$$



\paragraph{Задача 2. 2 балла.} Пассажир оставил вещи в 
автоматической камере хранения, а, когда пришел получать вещи, 
выяснилось, что он забыл номер. Он только помнил, что в номере 
были числа 23 и 37. Чтобы открыть камеру, нужно правильно 
набрать пятизначный номер. Какое наименьшее количество номеров 
нужно перебрать, чтобы наверняка открыть камеру? (Числа 23 и 37 
можно увидеть и в числе 237.)

\paragraph{\bf Решение:}
Рассмотрим возможные вариации кода: \\
В случае <<237>>:
\begin{enumerate}
\item 237**
\item *237*
\item **237
\end{enumerate}
В случае <<23 и 37>>:
\begin{enumerate}
\item 23*37
\item *2337
\item 2337*
\end{enumerate}
Последний случай надо будет домножать на 2, потому что числа 23 и 37
можно поменять местами.

У нас получается $ K_1 = 3 * 10^2 $ кодов для случая <<237>>, а также
$ K_2 = 3 * 10^1 $ кодов для случая <<23 и 37>>. А всего кодов, которые
стоит перебирать:
$$ K = K_1 + 2 * K_2 = 3 * 10^2 + 2 * 3 * 10^1 = 360 $$



\paragraph{Задача 3.} Двадцать школьников пришли в столовую. 
Сколькими различными способами они могут выстроиться в очередь, если:
а) (1 балл) Сережа обязательно стоит первым, Андрей -- четвертым, а Володя -- 
седьмым? б) (1 балл) Сережа, Андрей и Володя непременно
занимают первую, четвертую и седьмую позиции в очереди (в произвольном
порядке)? в) (1 балл) Сережа, Андрей и Володя занимают позиции с номерами 
$ i,j,k $ (опять-таки, в произвольном порядке), причем 
$ j-i = 3 $, $ k-j=3 $? 

\paragraph{\bf Решение:}
a) Если 3 школьника надо стоят на определенных местах, то осталось разместить
оставшихся в оставшиеся места:
$$ ANS_a = A_{17}^{17} = 17! = 355687428096000 $$
б) Теперь можно перемешать Андрея, Сережу и Володю на их местах. 
Пользуемся правилом умножения и предыдущим ответом:
$$ ANS_b = A_{3}^{3} * ANS_a = 2134124568576000 $$
в) Исходя из условия $ i \in [1; 14] $, то есть у нас 15 вариантов мест, для
знакомых нам школьников. И снова пользуемся правилом умножения:
$$ ANS_c = 15 * ANS_b = 32011868528640000 $$



\paragraph{Задача 4. 1 балл.} Сколькими способами можно выбрать четырех 
человек на четыре должности, если имеется девять кандидатов на эти 
должности? 

\paragraph{\bf Решение:}
Будем считать, что все должности разные, так как обратное не оговорено. Следовательно
речь идет о размещении 9 человек на 4 должности:
$$ ANS = A_{9}^{4} = \frac{9!}{5!} = 3024 $$


\paragraph{Задача 5.} Из класса, в котором учатся 28 человек, 
назначаются на дежурство в столовую 4 человека. а) (1 балл) Сколькими способами 
это можно сделать? б) (1 балл) Сколько существует способов набрать команду 
дежурных, в которую попадет ученик этого класса Коля Васин?

\paragraph{\bf Решение:}
а) Так как одна <<должность>> дежурного не отличается от другой, следовательно
разговор идет о 4-сочетании из 28 человек:
$$ ANS_a = CС_{28}^{4} = \frac{28!}{4! * (28 - 4)!} = 20475 $$
б) Так как один ученик определен, то речь пойдет о 3-сочетании из 27 человек:
$$ ANS_b = C_{27}^{3} = \frac{27!}{3! * (27 - 3)!} = 2925 $$

\paragraph{Задача 6. 1 балл.} В кондитерском магазине продавались 4 сорта 
пирожных: наполеоны, эклеры, песочные и слоеные. Сколькими способами 
можно купить 7 пирожных? 

\paragraph{\bf Решение:}


\end{document}


